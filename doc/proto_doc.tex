\documentclass{article}
\usepackage{graphicx} % Required for inserting images
\usepackage[T2A]{fontenc} % Кодировка шрифта
\usepackage[utf8]{inputenc} % Кодировка ввода
\usepackage[russian]{babel} % Языковые настройки
\usepackage{xcolor}
\usepackage{setspace}

\title{Zadanie2}
\author{Ильназ Батыршин}
\date{March 2024}

\begin{document}
\section*{Программирование искусственного интеллекта}
\begin{spacing} {1}
Программирование искусственного интеллекта (ИИ) - это область компьютерных наук, которая занимается созданием программ и систем, способных размышлять, решать проблемы и выполнять задачи, которые обычно требуют человеческого интеллекта. Развитие и применение ИИ имеет важное значение в множестве областей, таких как медицина, автомобильная промышленность, финансы и другие.
\end{spacing}
\par\bigskip\noindent\textcolor{blue} {Инновации в сфере ИИ}
\par\noindentСовременные разработки в области ИИ включают в себя машинное обучение, нейронные сети, обработку естественного языка, компьютерное зрение и многое другое. Они открывают новые возможности в решении сложных задач, улучшении медицинской диагностики, создании автономных автомобилей и повышении эффективности производства.

\begin{itemize} %маркер
    \item\textbf{Машинное обучение:} Область, в которой компьютеры могут обучаться на основе данных и делать прогнозы или принимать решения без явного программирования.

    \item\textbf{Нейронные сети:} Модель, инспирированная работой человеческого мозга, используемая для решения задач обработки информации.
\end{itemize}

\textit{Исскуственный интеллект} широко применяется в области анализа данных, автоматизации, медицинской диагностики, робототехники и многих других сферах.
\par\fcolorbox{cyan}{white} {Узнать больше о ИИ}.

\end{document}
